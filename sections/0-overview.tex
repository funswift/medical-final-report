\documentclass[../report]{subfiles}
\begin{document}

% 和文概要
\begin{jabstract}
本プロジェクトでは,3つのフィールドを調査し,そこから問題を見つけ,ICTを活用して解決する.
それにより地域・社会に貢献することを目標として活動を行っている.
開発手法はアジャイル開発手法を用いる.
素早くアプリを開発し,それに対するレビューを受けて問題解決の質をより高いものにしていく.

現在,我が国の認知症総患者数は2012年の時点で462万人に上り,今も増加の一途を辿っている.
認知症の発症には食事・運動・睡眠などによる生活習慣病が深く関わっており,健常者と比べて認知症患者の偏食が顕著であることが分かっている.
本グループは,日々の高齢者の食事データを撮影・蓄積し,食事のメニューの偏りを検出するシステムを構築することで,高齢者に食生活改善を促し,認知症予防に繋げられるのではないかと考えた.
これを実現するために,高齢者の食生活をICTを用いて記録・可視化できるようにする.
またそのデータを高齢者の家族や医療従事者と共有することで,システムが検出しきれなかった食事の偏りをヒトの目線から指摘できるようにする.
高齢者でも簡単に扱えるよう,少ない手順で日々の食事が撮影できるよう開発すること,日々の食事を蓄積し,医療選択時に必要となる情報を残して置けるようなシステムにすることを目的とした.

目標達成のために,本グループは高齢者でも簡単に食事を撮影することができるボックスと,撮影された食事の画像が高齢者の家族や医療従事者と共有できるWebサーバを開発した.
本システムでは,高齢者が日々の食事をボックスに入れることで,ボックス内のカメラが自動的に写真を撮影する.
その写真はWebサーバに送られ,高齢者の家族や医療従事者はブラウザを通して食事を確認することができ,またその食習慣についてメッセージを送信することができる.
高齢者はテレビを通して自身の食事の記録を見ることができ,また家族や医療従事者から貰ったメッセージを閲覧することができる.

\begin{jkeyword}
高齢者,認知症,医療,生活習慣,食習慣,ライフログ,カメラ
\end{jkeyword}
\bunseki{佐藤礼於}
\end{jabstract}


% 英文概要
\begin{eabstract}
In this project, we investigated three fields.
We find problems from them and solve them using ICT.
We are doing activities with the goal of contributing to the community and society by doing so.
We developed using agile development method.
We developed applications swiftly and receive reviews on it to make the quality of problem solving higher.

Today, the number of patients with dementia in our country has reached more than 4,620,000 as of 2012, and it is on an increasing trend year by year.
Lifestyle-related diseases such as diet, exercise, and sleep are deeply involved with the onset of dementia, and it is clear that uneven diet of dementia patients is more noticeable than healthy person.
This group think that the system can encourage elderly to improve dietary habit, and prevent dementia by collecting pictures of daily meal and developing system which can detect bias.
In order to do that, we make dietary habit of elderly recordable and visible using ICT.
Sharing that data with family and health care worker, it can detect bias which the system do not notice from person's perspective.
To use the system easier, so as to use the system with less labor, and save pictures of daily meal for informed consent.

To achieve this goal, this group developed the box which elderly can take the meal's picture easier, and the server which can save the pictures of elderly's meal and share them the family of elderly and health care worker.
The box take a picture automatically when elderly put meal in it.
The picture was send to the server, family of the elderly and health care worker can check eating habits of elderly through web browser, and can send the message about eating habits.
Elderly can check own eating habits through the TV, and receive messages from family and health care worker.

\begin{ekeyword}
Elderly, Dementia, Medical, Lifestyle, Dietary, Lifelog, Camera
\end{ekeyword}
\bunseki{佐藤礼於}
\end{eabstract}

\end{document}
