\documentclass[../report]{subfiles}
\setcounter{section}{7}
\begin{document}

本章では,今後の展望と学びについて説明する.
\bunseki{頼亜弥}

\section{今後の展望}
本グループの今後の展望として,主に三つ挙げられる.

一つ目は,ユーザーにあったおすすめ料理を表示させることである.
現状のシステムでは,ユーザの運動量や身長,体重などに合わせた料理を表示することはできない.
本グループはこういった機能を追加し,システムを改善していきたいと考えている.

二つ目は,ボックスを実用的な大きさ・形状に再設計することである.
本グループが作成したボックスに対して,自宅に置くには大きすぎるという意見を多数いただいた.
いただいた意見を受け止め,ボックスを再設計する必要性があると考えた.
補足として,いただいた意見の一部を以下に紹介する.

\begin{itemize}
    \item やっぱりボックスだと置く場所に困る,圧迫感がある.
    \item スタンド型が良い.
    \item 普段の傾向をつかむことやプロセスの透明化というのは医療選択する手段であるが,これをICTの力でアシストしてほしい.
\end{itemize}

三つ目は,家庭だけでなく,施設へのシステムの展開をすることである.施設では,家族は入居者の生活を見守ることが難しい場合が多い.本グループは,施設へのシステムの展開をすることでこのような問題を解消したいと考えている.
\bunseki{頼亜弥}

\section{学び}
本グループがこれまでのシステム開発の中で学んだこととして二つ挙げられる.

一つ目は,スケジュールの見通しやタスク管理である.
前期では,スケジュールの見通しやタスク管理が疎かになってしまい,ToDoを見落としてしまうことがあった.
後期からは,ガントチャートを作成したり,タスクに対する目標を立てるようにした.
その結果,ToDoを見落とすことが減り,作業の効率化に繋がった.

二つ目は,システム体験でいただくレビューの大切さである.
システムはグループ間だけで改善するのではなく,実際にユーザーの声を取り入れて改善を行うことで,自分たちだけでは発見できないユーザならではの視点で,システムに対するニーズが見えてくる.
\bunseki{頼亜弥}

\end{document}
